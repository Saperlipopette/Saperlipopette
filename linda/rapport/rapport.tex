\documentclass{report}

\usepackage[ansinew]{inputenc}
\usepackage[T1]{fontenc}
\usepackage{lmodern}
\usepackage[francais]{babel}
\usepackage{graphicx} % pour les images
\usepackage{amsmath} % les trois packages suivants sont pr les maths
\usepackage{amssymb}
\usepackage{mathrsfs}
\usepackage{listings} % pour l'affichage du code
\usepackage{color}

\lstset{ %
	language=Java,        % choix du langage
	basicstyle=\footnotesize,       % taille de la police du code
	numbers=left,                   % placer le numéro de chaque ligne à gauche (left) 
	numberstyle=\normalsize,        % taille de la police des numéros
	numbersep=7pt,            % distance entre le code et sa numérotation
	backgroundcolor=\color{white},
	commentstyle=\color{blue}
}
\makeatletter
\def\clap#1{\hbox to 0pt{\hss #1\hss}}%
\def\ligne#1{%
\hbox to \hsize{%
\vbox{\centering #1}}}%
\def\haut#1#2#3{%
\hbox to \hsize{%
\rlap{\vtop{\raggedright #1}}%
\hss
\clap{\vtop{\centering #2}}%
\hss
\llap{\vtop{\raggedleft #3}}}}%
\def\bas#1#2#3{%
\hbox to \hsize{%
\rlap{\vbox{\raggedright #1}}%
\hss
\clap{\vbox{\centering #2}}%
\hss
\llap{\vbox{\raggedleft #3}}}}%
\def\maketitle{%
\thispagestyle{empty}\vbox to \vsize{%
\haut{}{\@blurb}{}
\vfill
\vspace{1cm}
\begin{flushleft}
\usefont{OT1}{ptm}{m}{n}
\huge \@title
\end{flushleft}
\par
\hrule height 4pt
\par
\begin{flushright}
\usefont{OT1}{phv}{m}{n}
\Large \@author
\par
\end{flushright}
\vspace{1cm}
\vfill
\vfill
\bas{}{\@location, le \@date}{}
}%
\cleardoublepage
}
\def\date#1{\def\@date{#1}}
\def\author#1{\def\@author{#1}}
\def\title#1{\def\@title{#1}}
\def\location#1{\def\@location{#1}}
\def\blurb#1{\def\@blurb{#1}}
\date{\today}
\author{}
\title{}
\location{Toulouse}\blurb{}
\makeatother
\title{Rapport du projet de systèmes concurrents / intergiciels }
\author{Olivier Lienhard \\ Tom Lucas \\ Thibault Hilaire}
\location{Toulouse}
\blurb{%\blurb
École nationale supérieure d'électrotechnique, d'électronique, d'informatique, d'hydraulique et des télécommunications\\
\textbf{Filière Informatique et Mathématiques appliquées}\\[1em]
}% 



\begin{document}

\maketitle

\renewcommand{\contentsname}{Sommaire}
\renewcommand{\chaptername}{Partie}
\tableofcontents % pour que latex génére le sommaire

\chapter{Choix de la spécification libérale}

Explications des différents choix de la spécification libérale :
\begin{itemize}
 \item quand plusieurs tuples correspondent, take retourne le premier à avoir été écrit dans la mémoire (FIFO)
 \item quand plusieurs take sont en attente et qu'un dépôt peut en débloquer plusieurs, on débloque le premier take à avoir demandé (FIFO)
 \item quand des read et un take sont en attente, et qu'un dépôt peut les débloquer, on les débloque dans l'ordre de demande (FIFO)
 \item quand il y a un take et un callback enregistré pour le même motif, le take est prioritaire
\end{itemize}


\chapter{Version en mémoire partagée}

\section{Choix d'implémentation}

\chapter{Version client / mono-serveur}

\section{Choix d'implémentation}

\chapter{Version multi-serveurs}

\section{Choix d'implémentation}

\end{document}
