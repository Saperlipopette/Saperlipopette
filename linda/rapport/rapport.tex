\documentclass{report}

\usepackage[utf8]{inputenc}
\usepackage[T1]{fontenc}
\usepackage{lmodern}
\usepackage{graphicx} % pour les images
\usepackage{amsmath} % les trois packages suivants sont pr les maths
\usepackage{amssymb}
\usepackage{mathrsfs}
\usepackage{listings} % pour l'affichage du code
\usepackage{color}

\usepackage{color}
\definecolor{eclipseBlue}{RGB}{42,0.0,255}
\definecolor{eclipseGreen}{RGB}{63,127,95}
\definecolor{eclipsePurple}{RGB}{127,0,85}

\lstset{ %
	language=Java,        % choix du langage
	basicstyle=\footnotesize,       % taille de la police du code
	numbersep=7pt,            % distance entre le code et sa numérotation
	backgroundcolor=\color{white},
	commentstyle=\color{eclipseGreen},
  	keywordstyle=\color{eclipsePurple}, 
    stringstyle=\color{eclipseBlue},
    ndkeywordstyle=\color{eclipseBlue}, 
}
\makeatletter
\def\clap#1{\hbox to 0pt{\hss #1\hss}}%
\def\ligne#1{%
\hbox to \hsize{%
\vbox{\centering #1}}}%
\def\haut#1#2#3{%
\hbox to \hsize{%
\rlap{\vtop{\raggedright #1}}%
\hss
\clap{\vtop{\centering #2}}%
\hss
\llap{\vtop{\raggedleft #3}}}}%
\def\bas#1#2#3{%
\hbox to \hsize{%
\rlap{\vbox{\raggedright #1}}%
\hss
\clap{\vbox{\centering #2}}%
\hss
\llap{\vbox{\raggedleft #3}}}}%
\def\maketitle{%
\thispagestyle{empty}\vbox to \vsize{%
\haut{}{\@blurb}{}
\vfill
\vspace{1cm}
\begin{flushleft}
\usefont{OT1}{ptm}{m}{n}
\huge \@title
\end{flushleft}
\par
\hrule height 4pt
\par
\begin{flushright}
\usefont{OT1}{phv}{m}{n}
\Large \@author
\par
\end{flushright}
\vspace{1cm}
\vfill
\vfill
\bas{}{\@location, le \@date}{}
}%
\cleardoublepage
}
\def\date#1{\def\@date{#1}}
\def\author#1{\def\@author{#1}}
\def\title#1{\def\@title{#1}}
\def\location#1{\def\@location{#1}}
\def\blurb#1{\def\@blurb{#1}}
\date{13 Janvier 2015}
\author{}
\title{}
\location{Toulouse}\blurb{}
\makeatother
\title{Rapport du projet de systèmes concurrents / intergiciels }
\author{Olivier Lienhard \\ Tom Lucas \\ Thibault Hilaire}
\location{Toulouse}
\blurb{%\blurb
École nationale supérieure d'électrotechnique, d'électronique, d'informatique, d'hydraulique et des télécommunications\\
\textbf{Filière Informatique et Mathématiques appliquées}\\[1em]
}% 



\begin{document}

\maketitle

\renewcommand{\contentsname}{Sommaire}
\renewcommand{\chaptername}{Partie}
\tableofcontents % pour que latex génére le sommaire

\chapter{Introduction}

Ce projet s'attarde sur l'implémentation d'un modèle de coordination et communication nommé Linda.
Il est constitué par un espace de stockage d'objets appelés Tuple. Un Tuple est un n-uplet ordonné de 
valeurs comme ["voiture", 1]. Son template peut etre [String Integer] ou encore [Object Integer].
Une fois un tuple écrit il peut être pris (take) ou lu (read) par des processus. Notre implémentation
de Linda doit pouvoir gérer les demandes de plusieurs threads à la fois selon une spécification décrite
précédemment à l'implémentation. Les take consomment les tuples de l'espace, contrairement
aux read. L'avantage d'un tel modèle est que les différents processus n'ont pas conscience des autres,
ils communiquent vers l'espace partagé qui s'occupe de gérer les problèmes de concurrences. 

\chapter{Choix de la spécification libérale}

Explications des différents choix de la spécification libérale :
\begin{itemize}
 \item quand plusieurs tuples correspondent, take retourne le premier à avoir été écrit dans la mémoire (FIFO)
 \item quand plusieurs take sont en attente et qu'un dépôt peut en débloquer plusieurs, on débloque le premier take à avoir demandé (FIFO)
 \item quand des read et un take sont en attente, et qu'un dépôt peut les débloquer, on les débloque dans l'ordre de demande (FIFO)
 \item quand il y a un take et un callback enregistré pour le même motif, le take est prioritaire
\end{itemize}

\chapter{Version en mémoire partagée}

L'implantation mémoire partagée se situe dans le package linda.shm, et les tests associés dans le package linda.test.

\section{Choix d'implémentation}

Voici les différentes structures choisies pour respecter nos choix de spécification:
\begin{itemize}
 \item Collection<Tuple> tuples : pour sauvegarder nos tuples qui ont été écrits dans la mémoire, et les enlever lors d'un take
 \item	Map<Tuple, LinkedList<Integer>> MatchEnAttente : cette map va permettre de stocker tous les take/read bloquants 
 \item int id : un entier qui nous permettra de simuler notre FIFO, de savoir dans quels ordres les take/read ont été écrits dans notre map
 \item ArrayList<Condition> classe : un ensemble de conditions liées à un Lock, dont on associera chaque condition à un take/read bloquants
 \item Condition writeCondition : une condition qui va nous permettre de faire une priorité au signalé lors du réveil d'un take/read
 \item Boolean takeEffectue : un booléen permettant de vérifier, lors d'un write, si un take a été effectué
\end{itemize}


A partir de ceci, on peut expliquer l'algorithme de la fonction write :

Tout d'abord on ajoute le tuple dans la collection de tuples. Ensuite on récupère les templates dont le tuple ajouté correspond (i.e. tuple.matches(template)==true), ce qu'on appelle
les templatesCorrespondants. Puis tant que cette collection de templatesCorrespondants n'est pas vide ET qu'un take (sur ce tuple ajouté) n'a pas
été effectué (vérifiable sur notre booléen takeEffectue), on réveille le premier template correspondant (celui dont l'indice id est le plus bas).
Si un take est effectué, le booléen takeEffectue devient vrai, on sort de la boucle et on ne réveille pas les callback et la fonction write est finie.
Sinon, si aucun take n'est effectué et si on a parcouru toute la liste des templatesCorrespondants, on notifie les callback qui sont des observateurs.
Un tel algorithme garantit alors le respect de la spécification ci-dessus, qui est ensuite validée par les tests décrits ci-dessous.

\section{Tests}

Pour les tests, outre les tests donnés, des tests unitaires sont effectués pour chaque fonction pour chaque type de paramètre différent (notamment pour les callback). Ainsi une liste non exhaustive des tests unitaires principaux est :
\begin{itemize}
\item BasicTestAsyncCallbackReadFuture.java
\item BasicTestAsyncCallbackReadImmediate.java
\item BasicTestAsyncCallbackTakeFuture.java
\item BasicTestAsyncCallbackTakeImmediate.java
\item BasicTestCallbackReadFuture.java
\item BasicTestCallbackReadImmediate.java
\item BasicTestCallbackTakeFuture.java
\item BasicTestCallbackTakeImmediate.java
\item BasicTestRead.java
\end{itemize}
D'autres tests unitaires pour les fonctions tryRead, tryTake, TakeAll, ReadAll sont aussi définis. Des tests vérifiant quel les callback peuvent se réenregistrer sont aussi effectués.

Puis, pour vérifier que la spécification est bien respectée, nous avons effectué des tests pour vérifier chacun des quatre points, nommés respectivement dans l'ordre :
\begin{itemize}
\item BasicTestTakeSpec1.java
\item BasicTestTakeSpec2.java
\item BasicTestTakeReadSpec3.java
\item BasicTestTakeCallbackSpec4.java
\end{itemize}

Ces tests-là pour les principaux, ajoutés à d'autres tests, donnant tous des résultats cohérents, nous permettent de valider notre implémentation. 

\chapter{Version client / mono-serveur}

\section{Implémentation}

L'implantation mono-serveur se situe dans le package linda.mono.server, et les tests associés dans le package linda.testMonoServ.

\subsection{Serveur}
L'interface LindaServer hérite de java.rmi.remote , elle reprend les même fonctions que LindaCentralized, seule la signature de la méthode eventRegister est différente (voir section 3.1.3).
\paragraph{}La classe LindaServerImpl implémente cette interface et hérite de UnicastRemoteObject .Elle possède un attribut de type CentralizedLinda qui est réutilisé sans avoir été modifié.
Une fonction main() permet de lancer le serveur et de l'enregistrer ..... pour pouvoir être trouvé par les clients. 
\begin{lstlisting}
private CentralizedLinda linda;
	
public static void main(String[] args) {
	System.out.println("Demarrage du serveur");	
	String URL = "//localhost:4000/LindaServer";
	int port = 4000;
	try{
		LindaServer serveur = new LindaServerImpl();
		LocateRegistry.createRegistry(port);
		Naming.rebind(URL,serveur);
	}
	catch(Exception e){
		System.out.println("Une erreur s'est produite");
		e.printStackTrace();
	}	
}
\end{lstlisting}
\newpage
\paragraph{}A par pour les eventRegister (voir section 3.1.3), les autres fonctions du serveur consistent simplement en un appel aux fonctions équivalente du la version centralisé Linda du serveur:
\begin{lstlisting}
@Override
public Tuple tryTake(Tuple template) throws RemoteException {
	return linda.tryTake(template);
}
\end{lstlisting}
\subsection{Client}
La classe LindaClient implémente l'interface Linda.
Son constructeur prend en paramètre l'URL du serveur.
\begin{lstlisting}
 serveur = (LindaServer) Naming.lookup(URI);
\end{lstlisting}
Les fonctions de cette classe (à part EventRegister) consiste ainsi en l'appel des fonctions correspondantes du serveur.
\subsection{Le cas des EventRegister}
La principale difficulté de la version mono-serveur de Linda est l'implémentation de la méthode eventRegister car les callbacks pris en paramètre par la fonction eventRegister de la classe CentralizedLinda ne fonctionnent pas "à distance".
La solution à été de créer un callback spécial pour pouvoir faire cela, l'interface RemoteCallback hérite de java.rmi.remote et est implémentée par la classe RemoteCallbackImpl : son constructeur prend en paramètre un callback classique et sa méthode Call appelle la méthode callback de ce dernier.
\paragraph{}Le méthode EventRegister du serveur prend en paramètre un remoteCallback et crée un callback classique dont la fonction Call appelle celle du remoteCalback qu'elle passe en paramètre à la méthode EventRegister de CentralizedLinda : 
\begin{lstlisting}
@Override
public void eventRegister(eventMode mode, eventTiming timing,
	Tuple template, final RemoteCallback callback) 
	throws RemoteException {
	
	final class newCallback implements Callback {
		public void call(Tuple t) {
			try {
				callback.call(t);
			} catch (RemoteException e) {
				e.printStackTrace();
			}
		}
	}	
	linda.eventRegister(mode, timing, template, new newCallback() );
}
\end{lstlisting}
\paragraph{}La méthode EventRegister du client (qui implémente l'interface Linda) prend en paramètre un callback classique et crée un remoteCallback à partir de celui ci pour pouvoir appelé la méthode EventRegister du serveur:
\begin{lstlisting}
@Override
public void eventRegister(eventMode mode, eventTiming timing, 
					Tuple template, Callback callback) {
	try {
		RemoteCallback cb = new RemoteCallbackImpl(callback);
		serveur.eventRegister(mode, timing, template, cb);
	} catch (RemoteException e) {
		System.out.println("Erreur lors du eventRegister");
		e.printStackTrace();
	}
}
\end{lstlisting}
\section{Tests}
Tout les test effectués sur la version centralisé ont été effectués également sur la version mono-serveur, en donnant les même résultats.\newline
De plus cette version à été testé avec des "whiteboard" et ils fonctionnent parfaitement.

\chapter{Version client / multi-serveur}

Cette version n'a pas pu être implantée correctement, cela est dû à quelques difficultés et un léger manque de temps de notre part. Cependant, quelques fichiers d'implantation sont présents dans le package
linda.server, bien qu'ils n'aient pas aboutis.

\chapter{Conclusion}

	Ce projet a abordé des concepts relativement nouveaux. En particulier les eventRegister où il a
fallu s'approprié les sources fournies pour implementer ce système 'd'abonnement'. Une fois
la spécification libérale définie, l'implémentation des primitives reposait sur des connaissances
acquises. Il a fallu néanmoins prendre en compte certaines subtilités vu en cours comme la priorité
au signaleur/signalé. Pour la réalisation de la version en mémoire partagée ou mono-serveur , les eventRegister
ont demandé plus de temps puisqu'il fallait les prendres en compte dans le code précédemment ecrit. De 
plus, de nombreux tests ont du être fait pour s'assurer de leur bon fonctionnement ainsi que de la 
cohérence sur les priorités définies.
	Nous avons passé du temps sur la version multi-serveurs mais sans réussite. Nous étions partis sur
un système d'eventRegister où un serveur qui reçoit une requète s'abonne sur les différents serveurs.
La difficulté majeure de cette version est de gérer l'arrêt de ces abonnements lorsqu'un est trouvé.
C'est sur ce point que nous avons pas trouvé de solutions.
	Notre groupe a cependant compris la globalité du sujet. Nous avons avancé ensemble et tenté des
améliorations ( implantation multi-activité, nous pourrons aussi en discuter lors de l'oral bien que nous en ayons enlevé toute trace dans le code final ). Nous avons le sentiment de maîtriser les parties du
sujet traîtés et sommes d'accord sur les choix qui ont du être faits.



\end{document}
